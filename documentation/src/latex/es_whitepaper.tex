% EMPREINTE SOCIOMETRIQUE white paper

%----------------------------------------------------------------------------------------
%	PACKAGES AND OTHER DOCUMENT CONFIGURATIONS
%----------------------------------------------------------------------------------------

\documentclass[twoside,twocolumn]{article}

\usepackage{blindtext} % Package to generate dummy text throughout this template 

\usepackage[sc]{mathpazo} % Use the Palatino font
\usepackage[T1]{fontenc} % Use 8-bit encoding that has 256 glyphs
%\linespread{1.05} % Line spacing - Palatino needs more space between lines
\usepackage{microtype} % Slightly tweak font spacing for aesthetics
\usepackage{eufrak}
\usepackage{graphicx} % For \scalebox

\usepackage[english]{babel} % Language hyphenation and typographical rules

\usepackage[hmarginratio=1:1,top=32mm,columnsep=20pt]{geometry} % Document margins
\usepackage[hang, small,labelfont=bf,up,textfont=it,up]{caption} % Custom captions under/above floats in tables or figures
\usepackage{booktabs} % Horizontal rules in tables

\usepackage{lettrine} % The lettrine is the first enlarged letter at the beginning of the text

\usepackage{enumitem} % Customized lists
\setlist[itemize]{noitemsep} % Make itemize lists more compact

\usepackage{abstract} % Allows abstract customization
\renewcommand{\abstractnamefont}{\normalfont\bfseries} % Set the "Abstract" text to bold
\renewcommand{\abstracttextfont}{\normalfont\small\itshape} % Set the abstract itself to small italic text

\usepackage{titlesec} % Allows customization of titles
\renewcommand\thesection{\Roman{section}} % Roman numerals for the sections
\renewcommand\thesubsection{\arabic{subsection}} % roman numerals only for subsections
\titleformat{\section}[block]{\Large\scshape\centering}{\thesection.}{1em}{} % Change the look of the section titles
\titleformat{\subsection}[block]{\large\scshape}{\thesubsection.}{1em}{} % Change the look of the section titles

\usepackage{fancyhdr} % Headers and footers
\pagestyle{fancy} % All pages have headers and footers
\fancyhead{} % Blank out the default header
\fancyfoot{} % Blank out the default footer
\fancyhead[C]{Empreinte Sociométrique $\bullet$ Cyril Dever} % Custom header text
\fancyfoot[RO,LE]{\thepage} % Custom footer text
\setlength{\headheight}{14pt}

\usepackage{titling} % Customizing the title section

\usepackage{hyperref} % For hyperlinks in the PDF

\usepackage[symbol]{footmisc} % To use special character in footnote
\renewcommand{\thefootnote}{\arabic{footnote}}

\usepackage{outlines}
\usepackage[font=itshape]{quoting}

\usepackage[linesnumbered,ruled,vlined]{algorithm2e}
\SetKw{Continue}{continue}
\SetKw{KwBy}{by}

%----------------------------------------------------------------------------------------
%	FUNCTIONS
%----------------------------------------------------------------------------------------

\newcommand{\ceil}[1]{\left\lceil #1 \right\rceil}
\newcommand{\floor}[1]{\left\lfloor #1 \right\rfloor}
\newcommand{\bsfnote}{\textsuperscript{*}} % for reference to the base64 string note
\newcommand{\hexnote}{\textsuperscript{$\dagger$}} % for reference to the hex string note
\newcommand{\mod}[1]{\ \mathrm{mod}\ #1}

%----------------------------------------------------------------------------------------
%	LISTINGS
%----------------------------------------------------------------------------------------

\usepackage{amsthm}
\theoremstyle{definition}
\newtheorem{definition}{Definition}

\theoremstyle{remark}
\newtheorem*{remark}{Note}
\newtheorem*{recall}{Recall}

%----------------------------------------------------------------------------------------
%	FIGURES
%----------------------------------------------------------------------------------------

\usepackage{tikz}
\usepackage{caption}

\usetikzlibrary{shapes.geometric, arrows, calc, positioning}

\tikzstyle{startstop} = [rectangle, rounded corners, minimum width=3cm, minimum height=1cm,text centered, draw=black]
\tikzstyle{io} = [trapezium, trapezium left angle=70, trapezium right angle=110, minimum width=3cm, minimum height=1cm, text centered, text width=1.7cm, inner sep=0.4cm, draw=black]
\tikzstyle{process} = [rectangle, minimum width=3cm, minimum height=1cm, text centered, draw=black]
\tikzstyle{decision} = [diamond, minimum width=3cm, minimum height=1cm, text centered, inner sep=-0.1cm, draw=black]
\tikzstyle{arrow} = [thick,->,>=stealth]
\tikzset{XOR/.style={draw,circle,append after command={
        [shorten >=\pgflinewidth, shorten <=\pgflinewidth,]
        (\tikzlastnode.north) edge (\tikzlastnode.south)
        (\tikzlastnode.east) edge (\tikzlastnode.west)
        }
    }
}

%----------------------------------------------------------------------------------------
%	TITLE SECTION
%----------------------------------------------------------------------------------------

\usepackage[english]{datetime2}
\DTMsavedate{thedate}{2016-11-14}

\setlength{\droptitle}{-5\baselineskip} % Move the title up

\pretitle{\begin{center}\Large\bfseries}
\posttitle{\end{center}}
\title{Empreinte Sociométrique} % Title
\author{%
    \textsc{Cyril Dever}\\ % Name
    \normalsize Edgewhere \\ % Institution
}
% \date{\today} % Leave empty to omit a date
\date{\DTMusedate{thedate}}
\renewcommand{\maketitlehookd}{%
    \begin{abstract}
        \noindent Lorem ipsum ... % // TODO 
    \end{abstract}
}

%----------------------------------------------------------------------------------------

\begin{document}

% Print the title
\maketitle

%----------------------------------------------------------------------------------------
%	ARTICLE CONTENTS
%----------------------------------------------------------------------------------------

\section{Introduction}

\lettrine[nindent=0em,lines=3]{L}ike a tyre or a shoe leaving a distinctive mark on the ground, or a finger on a glass, each one of 
us leaves a trace the more specific the richer our social interactions are. In particular, the history of our contact data is every 
day more distinct to someone else's. Even before we move from our parents house, we start leaving personal trails (a first cell 
phone, a pseudo we use for a game, \dots) and, of course, our full civil status (names, date of birth, etc.).

Of course, we wouldn't want to share all these information to everyone. So, ensuring maximum security is obviously mandatory when 
it comes to manipulating personal data.

We describe a way to build such a safe footprint that we call \emph{Empreinte Sociométrique}\cite{empreinteSociometrique:cyd} and 
that is both totally secure, thanks to the use of strong pseudonymization techniques, and particularly effective.

Embedded in a QR Code, it could become an assistant to any identification device.

%----------------------------------------------------------------------------------------

\section{Formal Description}

\subsection{General Algorithm}

\begin{definition}[Source Data]
    \label{sourceData}
    A source data $\varsigma$ is the actual contact data we want to print in the \emph{Empreinte Sociométrique}, eg. \texttt{"Cyril"}.
    
    It is defined in the \emph{words} space: $\omega$.
\end{definition}

\begin{definition}[Data Type]
    \label{dataType}
    We define the data type $\tau$ as a code defining which kind of source data we are dealing with, eg. \texttt{"firstname"}.
    
    It is defined in a set of data types $\mathcal{T}$\footnote{see Table \ref{table:inputTypes} for available values in $\mathcal{T}$}.
\end{definition}

\begin{definition}[Input Data]
    \label{inputData}
    We define an input data $d_i$ as a tuple of data type and source data:
    \begin{small}
        \begin{equation}
            \label{eq:inputData}
                d_i := [d_i^\tau, d_i^\varsigma] \textrm{ with}\left\{
                    \begin{array}{l}
                        d_i^\tau \in \mathcal{T}, \textrm{the data type} \\ \\
                        d_i^\varsigma \in \omega, \textrm{the source data} \\
                    \end{array}
                \right.
        \end{equation}
    \end{small}
\end{definition}

\begin{definition}[Recombined Contact]
    \label{recombinedContact}
    A recombined contact is a final contact data potentially made out of different input data.

    For example, you can create a recombined \texttt{address4} by concatenating a \texttt{streetName} with a \texttt{streetNumber}.
\end{definition}

Let $\nu: \omega \to \omega$ be a normalization function that takes an input data and returns its normalized counterpart.

And let $\rho: \omega^n \to \omega$ be a recombination function that takes several input data to build a missing recombined contact.

Finally, let $\mathfrak{h}()$ be the cryptographic hashing function\footnote{set as a system parameter}, $\mathfrak{c}(msg,key)$ an 
encryption function and $\zeta()$ a compression algorithm.

Algorithm \ref{algo:es} describes the general steps to take that leads from a set of input data to its \emph{Empreinte Sociométrique}.
\begin{algorithm}
    \SetKwProg{throw}{throw}{}{}
    \KwIn{A vector $\textbf{d} := \{ d_1, d_2, \dots, d_n \}$ of input data, a key $K$}
    \KwOut{The \emph{Empreinte Sociométrique} or an error}
    \If{$\textbf{d} = \emptyset$}{
        \throw{empty input data set}{}
    }
    initialize the set of normalized data $\mathcal{D} \gets \emptyset$; \\
    \For{$i \gets 0$ \KwTo $n$ \KwBy $1$}{
        \If{$d_i^\tau \not\in \mathcal{T}$}{
            \Continue;
        }
        normalize input data: $d_{Norm} \gets \nu(d_i)$; \\
        \If{$d_{Norm} \neq \emptyset$}{
            $\mathcal{D} \gets d_{Norm}$; \\
        }
    }
    create the set of recombined contacts $\mathcal{R}$ from the normalized data:$$
        \mathcal{R} \gets \rho(\mathcal{D});
    $$ \\
    initialize the vector of ciphered contacts $\mathcal{C} \gets \emptyset$; \\
    \For{$i \gets 0$ \KwTo $|\mathcal{R}|$ \KwBy $1$}{
        $\mathcal{C} \gets \mathfrak{h}(\mathcal{R}_i)$; \\
    }
    initialize the sets of categorized contacts: $\mathcal{V}$ the variants, and $\overline{\mathcal{V}}$ the invariants; \\
    \For{$i \gets 0$ \KwTo $|\mathcal{C}|$ \KwBy $1$}{
        \eIf{$\mathcal{C}_i$ is invariant}{
            $\overline{\mathcal{V}} \gets \mathcal{C}_i$; \\
        }{
            $\mathcal{V} \gets \mathcal{C}_i$; \\
        }
    }
    build the corpus $c$ using $\mathcal{V}$ and $\overline{\mathcal{V}}$; \\
    encrypt it and compress it to build the \emph{Empreinte Sociométrique}:$$
        ES \gets \zeta \circ \mathfrak{c}(c, K);
    $$ \\
    \Return{$ES$};
    \caption{Overall construction}
    \label{algo:es}
\end{algorithm}

% \vfill\eject % To force break column if need be
% \tableofcontents % Uncomment to add a table of contents

%----------------------------------------------------------------------------------------
%	REFERENCE LIST
%----------------------------------------------------------------------------------------

\begin{thebibliography}{99} % Bibliography

\bibitem[1]{empreinteSociometrique:cyd}
Cyril Dever. \emph{Système de traitement d'une base de données personnelle par un opérateur extérieur}, patent pending \texttt{FR1905778}, 2019.

\end{thebibliography}

%----------------------------------------------------------------------------------------

\section*{Appendix}

\begin{table*}
    \centering
    \caption{Input data types}
    \begin{tabular*}{0.75\textwidth}{l|l||l}
        Code & Description & \textit{Examples} \\
        \hline \hline
        \texttt{gender} & Title or gender & \textit{M}, \textit{Mr.}, \textit{1}, \dots \\
        \hline
        \texttt{firstname} & First name or given name & \textit{John} \\
        \hline
        \texttt{middle} & Middle initials or other names & \textit{J.}, \textit{John} \\
        \hline
        \texttt{birthname} & Birth name & \textit{Kennedy} \\
        \hline
        \texttt{lastname} & Last name or married name & \textit{Kennedy} \\
        \hline
        \texttt{suffix} & Suffix & \textit{Jr.} \\
        \hline
        \texttt{birthdate} & Date of birth & \\
        \hline
        \texttt{birthplace} & Place of birth & \\
        \hline
        \texttt{addresses} & List of postal address & (see Table \ref{table:address}) \\
        \hline
        \texttt{aliases} & List of aliases & (see Table \ref{table:aliases}) \\
        \hline
        \texttt{emails} & List of e-mail addresses & (see Table \ref{table:emails}) \\
        \hline
        \texttt{ids} & List of official IDs & (see Table \ref{table:ids}) \\
        \hline
        \texttt{mobiles} & List of mobile phones & (see Table \ref{table:mobiles}) \\
        \hline
        \texttt{phones} & List of telephone numbers & (see Table \ref{table:phones}) \\
        \hline
        \texttt{socials} & List of social media & (see Table \ref{table:socials}) \\
        \hline
        \texttt{updated} & Unix timestamp of collect & \textit{1544529071}
        \label{table:inputTypes}
    \end{tabular*}
\end{table*}

\begin{table*}
    \centering
    \caption{Address data type}
    \begin{tabular*}{\textwidth}{l|l||l}
        Field & Definition & \textit{Possible values (or examples)} \\
        \hline \hline
        \texttt{type} & Address type & $\texttt{"birth"} \vee \texttt{"home"} \vee \texttt{"work"}$ \\
        \hline
        \texttt{address2} & Additional name & \textit{c/o Mme Dupont} \\
        \hline
        \texttt{address3} & Additional address & \textit{Apt. 123} \\
        \hline
        \texttt{address4} & Street number and name & \textit{1600 Pennsylvania Ave NW} \\
        \hline
        \texttt{address5} & PO Box or locality & \textit{BP 987} \\
        \hline
        \texttt{address6} & City and ZIP code & \textit{Washington, DC 20500} \\
        \hline
        \texttt{address7} & International destination & \textit{U.S.A.} \\
        \hline
        \texttt{city} & City & \textit{Washington} \\
        \hline
        \texttt{country} & Country & \textit{United States of America} \\
        \hline
        \texttt{fullAddress} & Full address & \textit{1600 Pennsylvania Ave NW, Washington, DC 20500} \\
        \hline
        \texttt{streetName} & Street name & \textit{Pennsylvania Ave NW} \\
        \hline
        \texttt{streetNumber} & Street number & \textit{1600} \\
        \hline
        \texttt{zip} & ZIP code & \textit{DC 20500} \\
        \hline
        \texttt{updated} & Time of collect & \textit{1544529071}
        \label{table:address}
    \end{tabular*}
\end{table*}

\begin{table*}
    \centering
    \caption{Alias data type}
    \begin{tabular*}{1.05\textwidth}{l|l||l}
        Field & Definition & \textit{Possible values (or examples)} \\
        \hline \hline
        \texttt{type} & Alias type & $\texttt{"commonname"} \vee \texttt{"identity"} \vee \texttt{"np"} \vee \texttt{"pn"} \vee \texttt{"pseudo"} \vee \texttt{"tnp"} \vee \texttt{"tpn"}$ \\
        \hline
        \texttt{value} & Alias value & \textit{John John} \\
        \hline
        \texttt{updated} & Time of collect & \textit{1544529071}
        \label{table:aliases}
    \end{tabular*}
\end{table*}

\begin{table*}
    \centering
    \caption{E-mail data type}
    \begin{tabular*}{0.75\textwidth}{l|l||l}
        Field & Definition & \textit{Possible values (or examples)} \\
        \hline \hline
        \texttt{type} & E-mail type & $\texttt{"business"} \vee \texttt{"personal"} \dots$ \\
        \hline
        \texttt{value} & E-mail address & \textit{john@john.com} \\
        \hline
        \texttt{isHash} & If is already hashed & $\texttt{true} \vee \texttt{false}$ \\
        \hline
        \texttt{engine} & Hashing algorithm & $\texttt{"blake2"} \vee \texttt{"md5"} \dots$ \\
        \hline
        \texttt{updated} & Time of collect & \textit{1544529071}
        \label{table:emails}
    \end{tabular*}
\end{table*}

\begin{table*}
    \centering
    \caption{ID data type}
    \begin{tabular*}{1.13\textwidth}{l|l||l}
        Field & Definition & \textit{Possible values (or examples)} \\
        \hline \hline
        \texttt{type} & ID type & $\texttt{"id"} \vee \texttt{"cb"} \vee \texttt{"passport"} \vee \texttt{"registration"} \vee \texttt{"serial"} \vee \texttt{"ss"} \vee \texttt{"udid"}$ \\
        \hline
        \texttt{value} & ID value & \textit{1234567890abcdef} \\
        \hline
        \texttt{isHash} & If is already hashed & $\texttt{true} \vee \texttt{false}$ \\
        \hline
        \texttt{engine} & Hashing algorithm & $\texttt{"blake2"} \vee \texttt{"md5"} \dots$ \\
        \hline
        \texttt{updated} & Time of collect & \textit{1544529071}
        \label{table:ids}
    \end{tabular*}
\end{table*}

\begin{table*}
    \centering
    \caption{Mobile phone data type}
    \begin{tabular*}{0.75\textwidth}{l|l||l}
        Field & Definition & \textit{Possible values (or examples)} \\
        \hline \hline
        \texttt{type} & Mobile phone type & $\texttt{"business"} \vee \texttt{"personal"} \dots$ \\
        \hline
        \texttt{value} & Mobile phone number & \textit{+33 (0) 623 456 789} \\
        \hline
        \texttt{isHash} & If is already hashed & $\texttt{true} \vee \texttt{false}$ \\
        \hline
        \texttt{engine} & Hashing algorithm & $\texttt{"blake2"} \vee \texttt{"md5"} \dots$ \\
        \hline
        \texttt{format} & Format & \texttt{"+dd (d) ddd ddd ddd"} \\
        \hline
        \texttt{updated} & Time of collect & \textit{1544529071}
        \label{table:mobiles}
    \end{tabular*}
\end{table*}

\begin{table*}
    \centering
    \caption{Telephone data type}
    \begin{tabular*}{0.75\textwidth}{l|l||l}
        Field & Definition & \textit{Possible values (or examples)} \\
        \hline \hline
        \texttt{type} & Telephone type & $\texttt{"business"} \vee \texttt{"personal"} \dots$ \\
        \hline
        \texttt{value} & Phone number & \textit{+33 (0) 123 456 789} \\
        \hline
        \texttt{isHash} & If is already hashed & $\texttt{true} \vee \texttt{false}$ \\
        \hline
        \texttt{engine} & Hashing algorithm & $\texttt{"blake2"} \vee \texttt{"md5"} \dots$ \\
        \hline
        \texttt{format} & Format & \texttt{"+dd (d) ddd ddd ddd"} \\
        \hline
        \texttt{updated} & Time of collect & \textit{1544529071}
        \label{table:phones}
    \end{tabular*}
\end{table*}

\begin{table*}
    \centering
    \caption{Social media data type}
    \begin{tabular*}{0.9\textwidth}{l|l||l}
        Field & Definition & \textit{Possible values (or examples)} \\
        \hline \hline
        \texttt{type} & Alias type & $\texttt{"facebook"} \vee \texttt{"linkedin"} \vee \texttt{"twitter"} \vee \texttt{"youtube"} \dots$ \\
        \hline
        \texttt{value} & Alias value & \textit{@john} \\
        \hline
        \texttt{updated} & Time of collect & \textit{1544529071}
        \label{table:socials}
    \end{tabular*}
\end{table*}

\end{document}