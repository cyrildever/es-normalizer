% EMPREINTE SOCIOMETRIQUE white paper

%----------------------------------------------------------------------------------------
%	PACKAGES AND OTHER DOCUMENT CONFIGURATIONS
%----------------------------------------------------------------------------------------

\documentclass[twoside,twocolumn]{article}

\usepackage{blindtext} % Package to generate dummy text throughout this template 

\usepackage[sc]{mathpazo} % Use the Palatino font
\usepackage[T1]{fontenc} % Use 8-bit encoding that has 256 glyphs
%\linespread{1.05} % Line spacing - Palatino needs more space between lines
\usepackage{microtype} % Slightly tweak font spacing for aesthetics
\usepackage{eufrak}
\usepackage{graphicx} % For \scalebox

\usepackage[english]{babel} % Language hyphenation and typographical rules

\usepackage[hmarginratio=1:1,top=32mm,columnsep=20pt]{geometry} % Document margins
\usepackage[hang, small,labelfont=bf,up,textfont=it,up]{caption} % Custom captions under/above floats in tables or figures
\usepackage{booktabs} % Horizontal rules in tables

\usepackage{lettrine} % The lettrine is the first enlarged letter at the beginning of the text

\usepackage{enumitem} % Customized lists
\setlist[itemize]{noitemsep} % Make itemize lists more compact

\usepackage{abstract} % Allows abstract customization
\renewcommand{\abstractnamefont}{\normalfont\bfseries} % Set the "Abstract" text to bold
\renewcommand{\abstracttextfont}{\normalfont\small\itshape} % Set the abstract itself to small italic text

\usepackage{titlesec} % Allows customization of titles
\renewcommand\thesection{\Roman{section}} % Roman numerals for the sections
\renewcommand\thesubsection{\arabic{subsection}} % roman numerals only for subsections
\titleformat{\section}[block]{\Large\scshape\centering}{\thesection.}{1em}{} % Change the look of the section titles
\titleformat{\subsection}[block]{\large\scshape}{\thesubsection.}{1em}{} % Change the look of the section titles

\usepackage{fancyhdr} % Headers and footers
\pagestyle{fancy} % All pages have headers and footers
\fancyhead{} % Blank out the default header
\fancyfoot{} % Blank out the default footer
\fancyhead[C]{Empreinte Sociométrique $\bullet$ Cyril Dever} % Custom header text
\fancyfoot[RO,LE]{\thepage} % Custom footer text
\setlength{\headheight}{14pt}

\usepackage{titling} % Customizing the title section

\usepackage{hyperref} % For hyperlinks in the PDF

\usepackage[symbol]{footmisc} % To use special character in footnote
\renewcommand{\thefootnote}{\arabic{footnote}}

\usepackage{outlines}
\usepackage[font=itshape]{quoting}

\usepackage[linesnumbered,ruled,vlined]{algorithm2e}
\SetKw{Continue}{continue}
\SetKw{KwBy}{by}

%----------------------------------------------------------------------------------------
%	FUNCTIONS
%----------------------------------------------------------------------------------------

\usepackage{amsmath}

\newcommand{\ceil}[1]{\left\lceil #1 \right\rceil}
\newcommand{\concat}{\mathbin{{+}\mspace{-8mu}{+}}}
\newcommand{\floor}[1]{\left\lfloor #1 \right\rfloor}
\newcommand{\bsfnote}{\textsuperscript{*}} % for reference to the base64 string note
\newcommand{\hexnote}{\textsuperscript{$\dagger$}} % for reference to the hex string note
\newcommand{\modulo}[1]{\ \mathrm{mod}\ #1}
\newcommand{\norm}[1]{\left\Vert#1\right\Vert}

%----------------------------------------------------------------------------------------
%	LISTINGS
%----------------------------------------------------------------------------------------

\usepackage{amsthm}
\theoremstyle{definition}
\newtheorem{definition}{Definition}
\newtheorem{example}{Example}

\theoremstyle{remark}
\newtheorem*{remark}{Note}
\newtheorem*{recall}{Recall}

%----------------------------------------------------------------------------------------
%	FIGURES
%----------------------------------------------------------------------------------------

\usepackage{tikz}
\usepackage{caption}

\usetikzlibrary{shapes.geometric, arrows, calc, positioning}

\tikzstyle{startstop} = [rectangle, rounded corners, minimum width=3cm, minimum height=1cm,text centered, draw=black]
\tikzstyle{io} = [trapezium, trapezium left angle=70, trapezium right angle=110, minimum width=3cm, minimum height=1cm, text centered, text width=1.7cm, inner sep=0.4cm, draw=black]
\tikzstyle{process} = [rectangle, minimum width=3cm, minimum height=1cm, text centered, draw=black]
\tikzstyle{decision} = [diamond, minimum width=3cm, minimum height=1cm, text centered, inner sep=-0.1cm, draw=black]
\tikzstyle{arrow} = [thick,->,>=stealth]
\tikzset{XOR/.style={draw,circle,append after command={
        [shorten >=\pgflinewidth, shorten <=\pgflinewidth,]
        (\tikzlastnode.north) edge (\tikzlastnode.south)
        (\tikzlastnode.east) edge (\tikzlastnode.west)
        }
    }
}

%----------------------------------------------------------------------------------------
%	TITLE SECTION
%----------------------------------------------------------------------------------------

\usepackage[english]{datetime2}
\DTMsavedate{thedate}{2016-11-14}

\setlength{\droptitle}{-5\baselineskip} % Move the title up

\pretitle{\begin{center}\Large\bfseries}
\posttitle{\end{center}}
\title{Empreinte Sociométrique} % Title
\author{%
    \textsc{Cyril Dever}\\ % Name
    \normalsize Edgewhere \\ % Institution
}
% \date{\today} % Leave empty to omit a date
\date{\DTMusedate{thedate}}
\renewcommand{\maketitlehookd}{%
    \begin{abstract}
        \noindent We describe an algorithm that takes contact data to represent the social footprint of a person in a secure and universal way. It 
        allows recognizing an identity without sharing or exposing any personal information to any potential eavesdropper. It could evolve with time 
        and may carry the whole contact history of a person, making it possible to verify if he or she matches with even very old data, all without 
        needing any kind of disclosure between parties. Last update: this technique is pending patent\footnotemark.
    \end{abstract}
}

%----------------------------------------------------------------------------------------

\begin{document}

% Print the title
\maketitle

{\renewcommand{\thefootnote}{\fnsymbol{footnote}} \footnotetext[1]{filed under registration number \texttt{FR1905778} at INPI on May 29, 2019}}

%----------------------------------------------------------------------------------------
%	ARTICLE CONTENTS
%----------------------------------------------------------------------------------------

\section{Introduction}

\lettrine[nindent=0em,lines=3]{L}ike a tyre or a shoe leaving a distinctive mark on the ground, or a finger on a glass, each one of us leaves a trace 
the more specific the richer our social interactions are. In particular, the history of our contact data is every day more distinct to someone else's. 
Even before we move from our parents house, we start leaving personal trails (a first cell phone, a pseudo we use for a game, \dots) and, of course, our 
full civil status (names, date of birth, etc.).

Of course, we wouldn't want to share all these information to everyone. So, ensuring maximum security is obviously mandatory when it comes to manipulating 
personal data.

We herein describe a way to build such a safe footprint that we call \emph{Empreinte Sociométrique}\cite{empreinteSociometrique:cyd} and that is both 
totally secure, thanks to the use of strong pseudonymization techniques, and particularly effective, that is it ensures that, even though two different 
\emph{Empreintes Sociométriques} don't look quite alike, they might be representing the same person, but no external stakeholder may ever know.

Embedded in a QR Code or used as is, it could become an assistant to any identification device or to further data manipulation, such as secure 
deduplication and anonymous enrichment, or even blind comparison of customer databases, all in full compliance with the latest regulations (GDPR, CCPA, 
\dots).

%----------------------------------------------------------------------------------------

\section{Formal Description}

\subsection{General Algorithm}

An \emph{Empreinte Sociométrique}\footnote{a literate translation being a \emph{Sociometric Imprint}} takes an input data characterized by its source and 
its type and operates several operations to form the final elements: its corpus and its signature concatenated into one long string.

\begin{definition}[Source Data]
    \label{sourceData}
    A source data $\varsigma$ is the actual contact data we want to print in the \emph{Empreinte Sociométrique}, eg. \texttt{"Cyril"}.
    
    It is defined in the \emph{words} space: $\omega$.
\end{definition}

\begin{definition}[Data Type]
    \label{dataType}
    We define the data type $\tau$ as a code defining which kind of source data we are dealing with, eg. \texttt{"firstname"}.
    
    It is defined in a set of data types $\mathcal{T}$ \footnote{see Table \ref{table:inputTypes} for available values in $\mathcal{T}$}.
\end{definition}

\begin{definition}[Input Data]
    \label{inputData}
    We define an input data $d_i$ as a tuple of data type and source data:
    \begin{small}
        \begin{equation}
            \label{eq:inputData}
                d_i := [d_i^\tau, d_i^\varsigma] \textrm{ with}\left\{
                    \begin{array}{l}
                        d_i^\tau \in \mathcal{T}, \textrm{the data type} \\ \\
                        d_i^\varsigma \in \omega, \textrm{the source data} \\
                    \end{array}
                \right.
        \end{equation}
    \end{small}
\end{definition}

\begin{definition}[Recombined Contact]
    \label{recombinedContact}
    A recombined contact is a final contact data potentially made out of different input data.

    For example, you can create a recombined \texttt{address4} by concatenating a \texttt{streetName} with a \texttt{streetNumber}.
\end{definition}

Let $\nu: \omega \to \omega$ be a normalization function that takes an input data and returns its normalized counterpart.

And let $\rho: \omega^n \to \omega$ be a recombination function that takes several input data to build a missing recombined contact.

Finally, let $\mathfrak{h}()$ be a cryptographic hashing function\footnote{set as a system parameter}, $\mathfrak{c}(msg,key)$ an encryption function 
and $\zeta()$ a compression algorithm.

\begin{algorithm}
    \SetKwProg{throw}{throw}{}{}
    \KwIn{A vector $\textbf{d} := \{ d_1, d_2, \dots, d_n \}$ of input data, a key $K$}
    \KwOut{The \emph{Empreinte Sociométrique} or an error}
    \If{$\textbf{d} = \emptyset$}{
        \throw{empty input data set}{}
    }
    initialize the set of normalized data $\mathcal{D} \gets \emptyset$; \\
    \For{$i \gets 0$ \KwTo $n$ \KwBy $1$}{
        \If{$d_i^\tau \not\in \mathcal{T}$}{
            \Continue;
        }
        normalize input data: $d_{Norm} \gets \nu(d_i)$; \\
        \If{$d_{Norm} \neq \emptyset$}{
            $\mathcal{D} \gets d_{Norm}$; \\
        }
    }
    create the set of recombined contacts $\mathcal{R}$ from the normalized data: $\mathcal{R} \gets \rho(\mathcal{D})$; \\
    initialize the vector of ciphered contacts $\mathcal{C} \gets \emptyset$; \\
    \For{$i \gets 0$ \KwTo $\norm{\mathcal{R}}$ \KwBy $1$}{
        $\mathcal{C} \gets \mathfrak{h}(\mathcal{R}_i)$; \\
    }
    initialize the sets of categorized contacts: $\mathcal{V}$ the variants, and $\overline{\mathcal{V}}$ the invariants; \\
    \For{$i \gets 0$ \KwTo $\norm{\mathcal{C}}$ \KwBy $1$}{
        \eIf{$\mathcal{C}_i$ is invariant}{
            $\overline{\mathcal{V}} \gets \mathcal{C}_i$; \\
        }{
            $\mathcal{V} \gets \mathcal{C}_i$; \\
        }
    }
    build the corpus $c$ using $\mathcal{V}$ and $\overline{\mathcal{V}}$; \\
    compress it and encrypt it partly to build the \emph{Empreinte Sociométrique}:$$
        ES \approx \zeta \circ \mathfrak{c}(c, K);
    $$ \\
    \Return{$ES$};
    \caption{Overall construction}
    \label{algo:es}
\end{algorithm}

Algorithm \ref{algo:es} describes the general steps to take that leads from a set of input data to its \emph{Empreinte Sociométrique}.

In a nutshell, the whole process could be summed up as follows:
\begin{itemize}
    \item Data handling;
    \item Normalization of contact data;
    \item Recombination to maximize endpoints;
    \item Ciphering of each endpoint;
    \item Organization of the corpus;
    \item Compression;
    \item Encryption;
    \item Signature and final packaging.
\end{itemize}

To ensure maximum security, the whole operation should take place on the data owner side.

\subsection{Step-by-Step Process}

As seen in Algorithm \ref{algo:es}, the very first step applied to an input data $d_i$ is to check whether it is a valid or not.

At this stage, an input data would be deemed invalid for two reasons:
\begin{itemize}
    \item The source data is empty: $d_i^\varsigma = \emptyset$;
    \item The data type doesn't exist: $d_i^\tau \not\in \mathcal{T}$.
\end{itemize}

Each invalid input data is discarded while each valid one is then applied the normalization function $\nu()$.

\subsubsection{Normalization}

The goal of such a normalization is to homogenize source data to make sure the subsequent operations handle equivalent data.

\begin{definition}[Equivalence]
    \label{equivalence}
    We speak of data equivalence when we cannot distinguish them from one another after normalization:
    \begin{equation}
        \label{eq:equivalence}
        \begin{array}{l}
            \forall x, y \in \omega: x \equiv y \\ \\
            \iff \left\{
                \begin{array}{l}
                    x^\tau = y^\tau \\ \\
                    \norm{\nu(x)} = \norm{\nu(y)} \\ \\
                    n \gets \norm{\nu(x)}, \forall i := [1..n]: \\
                        \quad \nu(x)[i] = \nu(y)[i] \\
                \end{array}
            \right. \\
            \end{array}
    \end{equation}

    In other words, two input data are said equivalent if they share the same data type and their normalized versions are identical.
\end{definition}

Normalization could be applied differently depending on the data type $d^\tau$.

The following are the transformations that might be applied to an input data source $d^\varsigma$:
\begin{itemize}
    \item Replacement of punctation and special characters by spaces, eg.$$
        \left.
            \begin{array}{r}
                \texttt{-} \\
                \texttt{.} \\
                \texttt{.} \\
                \texttt{*} \\
                etc. \\
            \end{array}
        \right\} \mapsto \backslash\texttt{s}
    $$
    \item Trim and extraction of excess spaces, eg.$$
        \backslash\texttt{s} \texttt{My} \backslash\texttt{s} \backslash\texttt{s} \texttt{data} 
            \mapsto \texttt{My} \backslash\texttt{s} \texttt{data} = \texttt{My data}
    $$
    \item Removal of accents, eg. \texttt{é} $\mapsto$ \texttt{e}, \texttt{ñ} $\mapsto$ \texttt{n};
    \item Capitalization, eg. \texttt{Cyril} $\mapsto$  \texttt{CYRIL};
    \item Dictionary substitution eg. $$
        \left.
            \begin{array}{r}
                \texttt{AVE} \\
                \texttt{AVENUE} \\
                \texttt{AVN} \\
                \texttt{AVNU}
            \end{array}
        \right\} \mapsto \texttt{AV} ~~~ 
        \left.
            \begin{array}{r}
                \texttt{F} \\
                \texttt{Madam} \\
                \texttt{Ms} \\
                \texttt{Mrs}
            \end{array}
        \right\} \mapsto \texttt{2} ~~~ etc.
    $$
\end{itemize}

In some case, normalization could transform the source data to a single space which in turn would lead to $d_i$ being discarded.

\begin{example}
    Let $$\left\{
        \begin{array}{l}
            d_1^\varsigma = \texttt{1600, Pennsylvania Avenue NW.} \\ \\
            d_2^\varsigma = \texttt{1600 PENNSYLVANIA AV NW} \\
        \end{array}
    \right.$$
    be two source data of the same data type:$$
        d_1^\tau = d_2^\tau = \texttt{address4},
    $$
    then we have:$$
        \nu(d_1^\varsigma) = \nu(d_2^\varsigma) = d_2^\varsigma \Longrightarrow d_1 \equiv d_2
    $$
    As $d_1$ and $d_2$ are equivalent, only one of them would be kept in the set of normalized data $\mathcal{D}$ to pass onto the recombination step.
\end{example}

\subsubsection{Recombination}

Once the data will be irreversibly ciphered into the final \emph{Empreinte Sociométrique}, we need to maximize the chance to get a match when using it, 
the goal of the recombination function $\rho$ is to provide as many representations as possible of the data.

Each data type may have dozens or even hundreds of possible combinations so we won't detail them here.
Instead, we will use an example.

\begin{example}
    Let the vector \textbf{d} be the input data:
    \begin{small} $$\begin{array}{l}
        \{ (\texttt{gender}, \texttt{Mr.}), (\texttt{firstname}, \texttt{John}), \\
        (\texttt{middle}, \texttt{F.}), (\texttt{lastname}, \texttt{Kennedy}), \\
        (\texttt{streetNumber}, \texttt{1600}), (\texttt{streetName}, \\
        \texttt{Pennsylvania Avenue}), (\texttt{city}, \texttt{Washington}), \\
        (\texttt{zip}, \texttt{DC 20500}) \}
    \end{array}$$ \end{small}

    Let $\mathcal{D} \gets \nu(\textbf{d})$  be a set of normalized data:
    \begin{small} $$\begin{array}{l}
        \{ \texttt{1}, \texttt{JOHN}, \texttt{F}, \texttt{KENNEDY}, \texttt{1600}, \\
        \texttt{PENNSYLVANIA AV}, \texttt{WASHINGTON}, \texttt{DC 20500} \} \\
    \end{array}$$ \end{small}

    Then $\mathcal{R} \gets \rho(\mathcal{D})$ might yield the following additional values:
    \begin{small} \begin{itemize}
        \item $~~\texttt{1 JOHN}$;
        \item $~~\texttt{JOHN F}$;
        \item $~~\texttt{1 JOHN F}$;
        \item $~~\texttt{JOHN KENNEDY}$;
        \item $~~\texttt{1 JOHN KENNEDY}$;
        \item $~~\texttt{JOHN F KENNEDY}$;
        \item $~~\texttt{1 JOHN F KENNEDY}$;
        \item $~~\texttt{1600 PENNSYLVANIA AV}$;
        \item $~~\texttt{WASHINGTON DC 20500}$;
        \item $\begin{array}{l} \texttt{1600 PENNSYLVANIA AV WASHINGTON} \\ \texttt{DC 20500}; \end{array}$
        \item $\begin{array}{l} \texttt{JOHN KENNEDY 1600 PENNSYLVANIA AV} \\\texttt{WASHINGTON DC 20500}; \end{array}$
        \item $\begin{array}{l} \texttt{1 JOHN KENNEDY 1600 PENNSYLVANIA} \\ \texttt{AV WASHINGTON DC 20500}; \end{array}$
        \item $\begin{array}{l} \texttt{JOHN F KENNEDY 1600 PENNSYLVANIA} \\ \texttt{AV WASHINGTON DC 20500}; \end{array}$
        \item $\begin{array}{l} \texttt{1 JOHN F KENNEDY 1600 PENNSYLVANIA} \\ \texttt{AV WASHINGTON DC 20500}. \end{array}$
    \end{itemize} \end{small}
    $\Longrightarrow \norm{\mathcal{R}} = 22$

    Once ciphered, these 22 recombined contacts along with their respective output codes $\Omega$ will form the basis of the corpus.
\end{example}

\begin{definition}[Output Code]
    \label{outputCode}
    An output code $\Omega$ is a code defining the specific type of a recombined contact. It is either a special version of the input type code or the 
    concatenation of its different parts.

    A complete list of output types can be found in Table \ref{table:outputCodes}. Each code might be combined with one or several others in order to 
    form a new output type.
\end{definition}

\subsubsection{Ciphering}

As we want to take advantage of the very characteristics of cryptographic hashing, ie. making an input its unique yet irreversible fixed-length image, 
and as we want it to be both secure and widely spread to devices, the cipher $\mathfrak{h}()$ to apply to each item of $\mathcal{R}$ should be a 
well-known well-tested cryptographic hashing function:
\begin{equation}
    \label{eq:ciphering}
    \forall i \in [1..\norm{\mathcal{R}}]: \mathcal{C}_i \gets \mathfrak{h}(\mathcal{R}_i)
\end{equation}

\begin{remark}
    Because of the subsequent steps of our process, the strength of $\mathfrak{h}()$ is actually not that important, so it could be an algorithm as 
    trivial as \texttt{MD-5} or a more secure one such as \texttt{Blake-2} or \texttt{Keccak}\footnote{Short list available on Wikipedia: 
    \url{https://bit.ly/3bDpGxd}}.

    The only thing mandatory is to set it once and for all to be sure any future utilization will use the same cipher.
\end{remark}

Once the ciphering stage finished, we have all the necessary components to build the first part of the \emph{Empreinte Sociométrique}: its corpus.

\subsubsection{Corpus}

To build the corpus, we first need to create the set of all imprints $\mathcal{E}$ from $\mathcal{C}$.

\begin{definition}[Imprint]
    \label{imprint}
    We define an imprint\footnote{Empreinte in French} $\varepsilon$ by the following tuple of information:
    \begin{itemize}
        \item Its general type code, eg. $\varepsilon^\tau = \texttt{ad}$ for an address type;
        \item The ciphered recombined normalized item $\varepsilon^c$;
        \item A timestamp of collect (or insertion) in the system, eg. $\varepsilon^t = \texttt{1544529071}$;
        \item Its specific short code $\varepsilon^{\overline{s}}$.
    \end{itemize}

    The complete list of type codes are given in Table \ref{table:empreintes}.

    It is easy to prove that each imprint is as safe as the functions used to build $\varepsilon^c$.
    And when we say "safe" it is by means of recovering the original data source (which doesn't exist anymore at this point).
\end{definition}

\begin{definition}[Short Code]
    \label{shortCode}
    A short code $\overline{s}$ is a special 4-characters hash defining the specific type of imprint.

    Algorithm \ref{algo:shortCode} describes the construction of a short code using the output field type $\Omega$.
    It uses a 62-characters basis $\mathcal{B}$ using the three following sequences in that order:
    \begin{itemize}
        \item The 10 figures: $1234567890$;
        \item The 26 lower-case alphabet;
        \item The 26 upper-case alphabet.
    \end{itemize}

    Let $idx(char)$ be a function that returns the index number (starting at $1$) of the passed character in $\mathcal{B}$.
    \begin{algorithm}
        \KwIn{$\Omega$}
        \KwOut{The corresponding short code}
        initialize the sum of $\Omega$ characters on that basis: $s \gets 0$; \\
        \For{$i \gets 1$ \KwTo $\norm{\Omega}$}{
            $s \gets s + idx(\Omega_i)$;
        }
        make it an hexadecimal string representation: $s_{16} := (s)_{16}$; \\
        hash $\Omega$ with \texttt{MD-5} hashing function: $h := \texttt{md5}(\Omega)$; \\
        extract and concatenate the first two characters of $s_{16}$ and $h$: $$
            \overline{s} \gets s_{16}[0] \concat s_{16}[1] \concat h[0] \concat h[1]
        $$ \\
        \Return{$\overline{s}$};
        \caption{Short code $\overline{s}$}
        \label{algo:shortCode}
    \end{algorithm}
    
    For example, for $\Omega = \texttt{email}$, we have: $\overline{s}(\Omega) := \texttt{0c5a}$
\end{definition}

\begin{definition}[Variance]
    \label{variantInvariant}
    A invariant contact data is a data that is not meant to change in a lifetime.
    For example, the date of birth is invariant.
    There is a limited set of variant contact data.

    On the other hand, a variant data may change in time (for example, a postal address where one lives).
    There is an unlimited list of invariant contact data. An \emph{Empreinte Sociométrique} may even carry multiple of the same types (for example, 
    if one has moved a lot, he should have as many invariant addresses as places he is been living).
\end{definition}

\begin{definition}[Corpus]
    \label{corpus}
    The corpus of an \emph{Empreinte Sociométrique} is a special concatenation of all the imprints built from the input data using a secret token $T_k$.
    It is made of variant and invariant imprints.

    Algorithm \ref{algo:corpus} describes its construction.

    Let \texttt{BLOCK\_SEP} and \texttt{SEC\_SEP} be two special string separators \footnote{In our current impmlementation, we use:
        \begin{itemize}
            \item The pipe character (\texttt{|}) as \texttt{BLOCK\_SEP};
            \item The percent character (\texttt{\%}) as \texttt{SEC\_SEP}.
        \end{itemize}
    }, \texttt{PREFIX} a system parameter setting a cutting index, and $cut(str, at)$ the cutting string utility.

    Finally, let \texttt{V\_HASH} be the hashing algorithm used.
    \begin{algorithm}
        \KwIn{The set of imprints $\mathcal{E}$, a secret token $T_k$}
        \KwOut{The corpus string}
        split the variant from the invariant imprints: $\mathcal{V}, \overline{\mathcal{V}} := \mathcal{E}$; \\
        initialize the invariants string: $\overline{v} \gets 0$; \\
        \For{$i \gets 1$ \KwTo $\norm{\overline{\mathcal{V}}}$ \KwBy $1$}{
            $\overline{v} \gets \overline{v} + \overline{\mathcal{V}}_i$; \\
            \If{$i < \norm{\overline{\mathcal{V}}}$}{
                $\overline{v} \gets \overline{v} \concat \texttt{BLOCK\_SEP}$; \\
            }
        }
        initialize the variants string: $v \gets 0$; \\
        \For{$i \gets 1$ \KwTo $\norm{\mathcal{V}}$ \KwBy $1$}{
            $v \gets v + \mathcal{V}_i$; \\
            \If{$i < \norm{\mathcal{V}}$}{
                $v \gets v \concat \texttt{BLOCK\_SEP}$; \\
            }
        }
        concat the core corpus: $c \gets \overline{v} \concat \texttt{SEC\_SEP} \concat v$; \\
        compress it: $C \gets \zeta(c)$; \\
        separate prefix $p^-$ from suffix $s^+$ at \texttt{PREFIX}: $p^-, s^+ := cut(C, \texttt{PREFIX})$; \\
        encrypt the prefix with the token: $\mathfrak{e} \gets \mathfrak{c}(p^-, T_k)$; \\
        get the length of the encrypted prefix: $l \gets \norm{\mathfrak{e}}$; \\
        set the version short code: $\upsilon \gets \overline{s}(\texttt{V\_HASH})$; \\
        build the full corpus $\chi$:\begin{equation}
            \chi \gets s^+ \concat \mathfrak{e} \concat \texttt{SEC\_SEP} \concat l \concat \upsilon
        \end{equation} \\
        hash the corpus with \texttt{MD-5} to get a 32-characters pseudo-random head string: $H \gets \texttt{md5}(\chi)$; \\
        \Return{$H \concat \chi$};
        \caption{Corpus construction}
        \label{algo:corpus}
    \end{algorithm}
\end{definition}

\subsubsection{Signature}

Lorem ipsum ...

\subsubsection{Output}

Lorem ipsum ...

% \vfill\eject % To force break column if need be
\tableofcontents % Uncomment to add a table of contents

%----------------------------------------------------------------------------------------
%	REFERENCE LIST
%----------------------------------------------------------------------------------------

\begin{thebibliography}{99} % Bibliography

\bibitem[1]{empreinteSociometrique:cyd}
Cyril Dever. \emph{Système de traitement d'une base de données personnelle par un opérateur extérieur}, patent pending \texttt{FR1905778}, 2019.

\end{thebibliography}

%----------------------------------------------------------------------------------------

\onecolumn
\section*{Appendix}

\begin{table}[htb]
    \centering
    \caption{Input data types}
    \begin{tabular*}{0.75\textwidth}{l|l||l}
        Code & Description & \textit{Examples} \\
        \hline \hline
        \texttt{gender} & Title or gender & \textit{M}, \textit{Mr.}, \textit{1}, \dots \\
        \hline
        \texttt{firstname} & First name or given name & \textit{John} \\
        \hline
        \texttt{middle} & Middle initials or other names & \textit{J.}, \textit{John} \\
        \hline
        \texttt{birthname} & Birth name & \textit{Kennedy} \\
        \hline
        \texttt{lastname} & Last name or married name & \textit{Kennedy} \\
        \hline
        \texttt{suffix} & Suffix & \textit{Jr.} \\
        \hline
        \texttt{birthdate} & Date of birth & \\
        \hline
        \texttt{birthplace} & Place of birth & \\
        \hline
        \texttt{addresses} & List of postal address & (see Table \ref{table:address}) \\
        \hline
        \texttt{aliases} & List of aliases & (see Table \ref{table:aliases}) \\
        \hline
        \texttt{emails} & List of e-mail addresses & (see Table \ref{table:emails}) \\
        \hline
        \texttt{ids} & List of official IDs & (see Table \ref{table:ids}) \\
        \hline
        \texttt{mobiles} & List of mobile phones & (see Table \ref{table:mobiles}) \\
        \hline
        \texttt{phones} & List of telephone numbers & (see Table \ref{table:phones}) \\
        \hline
        \texttt{socials} & List of social media & (see Table \ref{table:socials}) \\
        \hline
        \texttt{updated} & Unix timestamp of collect & \textit{1544529071}
        \label{table:inputTypes}
    \end{tabular*}
\end{table}

\begin{table}[htb]
    \centering
    \caption{Address data type}
    \begin{tabular*}{\textwidth}{l|l||l}
        Field & Definition & \textit{Possible values (or examples)} \\
        \hline \hline
        \texttt{type} & Address type & $\texttt{"birth"} \vee \texttt{"home"} \vee \texttt{"work"}$ \\
        \hline
        \texttt{address2} & Additional name & \textit{c/o Mme Dupont} \\
        \hline
        \texttt{address3} & Additional address & \textit{Apt. 123} \\
        \hline
        \texttt{address4} & Street number and name & \textit{1600 Pennsylvania Ave NW} \\
        \hline
        \texttt{address5} & PO Box or locality & \textit{BP 987} \\
        \hline
        \texttt{address6} & City and ZIP code & \textit{Washington, DC 20500} \\
        \hline
        \texttt{address7} & International destination & \textit{U.S.A.} \\
        \hline
        \texttt{city} & City & \textit{Washington} \\
        \hline
        \texttt{country} & Country & \textit{United States of America} \\
        \hline
        \texttt{fullAddress} & Full address & \textit{1600 Pennsylvania Ave NW, Washington, DC 20500} \\
        \hline
        \texttt{streetName} & Street name & \textit{Pennsylvania Ave NW} \\
        \hline
        \texttt{streetNumber} & Street number & \textit{1600} \\
        \hline
        \texttt{zip} & ZIP code & \textit{DC 20500} \\
        \hline
        \texttt{updated} & Time of collect & \textit{1544529071}
        \label{table:address}
    \end{tabular*}
\end{table}

\begin{table}[htb]
    \centering
    \caption{Alias data type}
    \begin{tabular*}{1.05\textwidth}{l|l||l}
        Field & Definition & \textit{Possible values (or examples)} \\
        \hline \hline
        \texttt{type} & Alias type & $\texttt{"commonname"} \vee \texttt{"identity"} \vee \texttt{"np"} \vee \texttt{"pn"} \vee \texttt{"pseudo"} 
            \vee \texttt{"tnp"} \vee \texttt{"tpn"}$ \\
        \hline
        \texttt{value} & Alias value & \textit{John John} \\
        \hline
        \texttt{updated} & Time of collect & \textit{1544529071}
        \label{table:aliases}
    \end{tabular*}
\end{table}

\begin{table}[htb]
    \centering
    \caption{E-mail data type}
    \begin{tabular*}{0.75\textwidth}{l|l||l}
        Field & Definition & \textit{Possible values (or examples)} \\
        \hline \hline
        \texttt{type} & E-mail type & $\texttt{"business"} \vee \texttt{"personal"} \dots$ \\
        \hline
        \texttt{value} & E-mail address & \textit{john@john.com} \\
        \hline
        \texttt{isHash} & If is already hashed & $\texttt{true} \vee \texttt{false}$ \\
        \hline
        \texttt{engine} & Hashing algorithm & $\texttt{"blake2"} \vee \texttt{"md5"} \dots$ \\
        \hline
        \texttt{updated} & Time of collect & \textit{1544529071}
        \label{table:emails}
    \end{tabular*}
\end{table}

\begin{table}[htb]
    \centering
    \caption{ID data type}
    \begin{tabular*}{1.13\textwidth}{l|l||l}
        Field & Definition & \textit{Possible values (or examples)} \\
        \hline \hline
        \texttt{type} & ID type & $\texttt{"id"} \vee \texttt{"cb"} \vee \texttt{"passport"} \vee \texttt{"registration"} \vee \texttt{"serial"} \vee 
            \texttt{"ss"} \vee \texttt{"udid"}$ \\
        \hline
        \texttt{value} & ID value & \textit{1234567890abcdef} \\
        \hline
        \texttt{isHash} & If is already hashed & $\texttt{true} \vee \texttt{false}$ \\
        \hline
        \texttt{engine} & Hashing algorithm & $\texttt{"blake2"} \vee \texttt{"md5"} \dots$ \\
        \hline
        \texttt{updated} & Time of collect & \textit{1544529071}
        \label{table:ids}
    \end{tabular*}
\end{table}

\begin{table}[htb]
    \centering
    \caption{Mobile phone data type}
    \begin{tabular*}{0.75\textwidth}{l|l||l}
        Field & Definition & \textit{Possible values (or examples)} \\
        \hline \hline
        \texttt{type} & Mobile phone type & $\texttt{"business"} \vee \texttt{"personal"} \dots$ \\
        \hline
        \texttt{value} & Mobile phone number & \textit{+33 (0) 623 456 789} \\
        \hline
        \texttt{isHash} & If is already hashed & $\texttt{true} \vee \texttt{false}$ \\
        \hline
        \texttt{engine} & Hashing algorithm & $\texttt{"blake2"} \vee \texttt{"md5"} \dots$ \\
        \hline
        \texttt{format} & Format & \texttt{"+dd (d) ddd ddd ddd"} \\
        \hline
        \texttt{updated} & Time of collect & \textit{1544529071}
        \label{table:mobiles}
    \end{tabular*}
\end{table}

\begin{table}[htb]
    \centering
    \caption{Telephone data type}
    \begin{tabular*}{0.75\textwidth}{l|l||l}
        Field & Definition & \textit{Possible values (or examples)} \\
        \hline \hline
        \texttt{type} & Telephone type & $\texttt{"business"} \vee \texttt{"personal"} \dots$ \\
        \hline
        \texttt{value} & Phone number & \textit{+33 (0) 123 456 789} \\
        \hline
        \texttt{isHash} & If is already hashed & $\texttt{true} \vee \texttt{false}$ \\
        \hline
        \texttt{engine} & Hashing algorithm & $\texttt{"blake2"} \vee \texttt{"md5"} \dots$ \\
        \hline
        \texttt{format} & Format & \texttt{"+dd (d) ddd ddd ddd"} \\
        \hline
        \texttt{updated} & Time of collect & \textit{1544529071}
        \label{table:phones}
    \end{tabular*}
\end{table}

\begin{table}[htb]
    \centering
    \caption{Social media data type}
    \begin{tabular*}{0.9\textwidth}{l|l||l}
        Field & Definition & \textit{Possible values (or examples)} \\
        \hline \hline
        \texttt{type} & Alias type & $\texttt{"facebook"} \vee \texttt{"linkedin"} \vee \texttt{"twitter"} \vee \texttt{"youtube"} \dots$ \\
        \hline
        \texttt{value} & Alias value & \textit{@john} \\
        \hline
        \texttt{updated} & Time of collect & \textit{1544529071}
        \label{table:socials}
    \end{tabular*}
\end{table}

\begin{table}[htb]
    \centering
    \caption{List of base output codes}
    \begin{tabular*}{0.6\textwidth}{l|l}
        Code & Description \\
        \hline \hline
        \texttt{adresse2} & Postal address line 2 \\
        \hline
        \texttt{adresse3} & Postal address line 3 \\
        \hline
        \texttt{adresse4} & Postal address line 4 \\
        \hline
        \texttt{adresse5} & Postal address line 5 \\
        \hline
        \texttt{adresse} & Full postal address \\
        \hline
        \texttt{civ} & Title \\
        \hline
        \texttt{cp} & Postal (or ZIP) code \\
        \hline
        \texttt{ddn} & Date of birth \\
        \hline
        \texttt{dpt} & Department, eg.\texttt{75} \\
        \hline
        \texttt{dv} & Department and city, eg. \texttt{75 PARIS} \\
        \hline
        \texttt{email} & E-mail address \\
        \hline
        \texttt{facebook} & Facebook ID \\
        \hline
        \texttt{instagram} & Instagram ID \\
        \hline
        \texttt{linkedin} & LinkedIn ID \\
        \hline
        \texttt{pinterest} & Pinterest ID \\
        \hline
        \texttt{snapchat} & Snapchat ID \\
        \hline
        \texttt{twitter} & Twitter ID or screen name \\
        \hline
        \texttt{youtube} & Youtube ID \\
        \hline
        \texttt{in} & Initials and last name \\
        \hline
        \texttt{ini} & Initials \\
        \hline
        \texttt{middle\_ini} & Middle initial(s) \\
        \hline
        \texttt{mob} & Mobile phone number \\
        \hline
        \texttt{ni} & Last name and initials \\
        \hline
        \texttt{nom} & Last (or birth) name \\
        \hline
        \texttt{np} & Last and first names \\
        \hline
        \texttt{no} & Street number \\
        \hline
        \texttt{pays} & Country \\
        \hline
        \texttt{pn} & First and last names \\
        \hline
        \texttt{prenom} & First (or given) name \\
        \hline
        \texttt{middle} & Middle names \\
        \hline
        \texttt{raisonsociale} & Corporate name \\
        \hline
        \texttt{tel} & Telephone / landline number \\
        \hline
        \texttt{tin} & Title + Initials + Last name \\
        \hline
        \texttt{tn} & Title and last name \\
        \hline
        \texttt{tni} & Title + Last name + Initials \\
        \hline
        \texttt{tnp} & Title + Last name + First name \\
        \hline
        \texttt{tpn} & Title + First name + Last name \\
        \hline
        \texttt{ville} & City \\
        \hline
        \texttt{voie} & Street name
        \label{table:outputCodes}
    \end{tabular*}
\end{table}

\begin{table}[htb]
    \centering
    \caption{List of imprint type codes}
    \begin{tabular*}{0.35\textwidth}{l|c}
        Type & Code \\
        \hline \hline
        Address & \texttt{ad} \\
        \hline
        Alias & \texttt{al} \\
        \hline
        Birth date & \texttt{4d} \\
        \hline
        Birth name & \texttt{2n} \\
        \hline
        Birth place & \texttt{5p} \\
        \hline
        Country & \texttt{8c} \\
        \hline
        E-mail & \texttt{ml} \\
        \hline
        Facebook & \texttt{fb} \\
        \hline
        First name & \texttt{3n} \\
        \hline
        Gender & \texttt{1x} \\
        \hline
        Instagram & \texttt{ig} \\
        \hline
        LinkedIn & \texttt{li} \\
        \hline
        Marital name & \texttt{mn} \\
        \hline
        Middle names & \texttt{7m} \\
        \hline
        Mobile phone & \texttt{mb} \\
        \hline
        Pinterest & \texttt{pi} \\
        \hline
        Registration & \texttt{rg} \\
        \hline
        Snapchat & \texttt{sn} \\
        \hline
        Social security number & \texttt{6s} \\
        \hline
        Telephone number & \texttt{tl} \\
        \hline
        Twitter & \texttt{tw} \\
        \hline
        Youtube & \texttt{yt}
        \label{table:empreintes}
    \end{tabular*}
\end{table}


\end{document}